\section{Historique}
\subsection{1950}
\subsection{1970}
\subsection{1980}
\begin{frame}{Historique}
    \begin{itemize}[<+-|alert@+>]
        \myitem 1950
        \myitem 1970
        \myitem 1980
        \myitem à partir de 1980
        \myitem récent
    \end{itemize}

    \only<1>{
    \begin{exampleblock}{Naissance du mot}
        le $ mot $ de IA est née dans les années 1950 avec l'objectif de faire
        produire des tâches humaines par des machines mimant l'activité du
        cerveau.
        %que ce soit par des algorithmes des
        %classification/reseux des neurons artificiels
        % image de alan torine dans reunis
    \end{exampleblock}
    }

    \only<2>{
    \begin{exampleblock} {Création des algorithmes}
        les scientifique ont pris fin à création de tous les algorithmes
        d'intelligence artificiels, mais la puisance des ordinateurs sont très
        faibles
    \end{exampleblock}
    }

    \only<3>{ \begin{exampleblock}{Mycin}
        le développement du premier applications qui utilise le IA dans le
        domaine de medcine, qui capable de reproduire les mécanismes cognitifs
        d'un expert, Les plus célèbres, Mycin (identification d'infections
        bactériennes) s'appuient sur l'ensemble des connaissances médicales
        dans un domaine donné et une formalisation des raisonnements des
        spécialistes qui lient ces connaissances entre elles pour aboutir à un
        diagnostic.
    \end{exampleblock}
    }

    \only<4>{
    \begin{exampleblock}{Réseaux des neurones artificielle}
        développement des réseaux de neurones artificiels, grâce à l'augmentation
        de puissance des ordinateurs et à l'accumulation des gigantesques quantités
        de données(big data).
    \end{exampleblock}
    }

    \only<5>{
    \begin{exampleblock}{Google}
        Google AI mis au point une IA qui prédit le cancer du poumon avec
        $94,4~\%$ de réussite. Ces procédures permettent aussi d'éviter des
        tests invasifs comme des biopsies. L'IA apporte également une aide à
        la prescription, par exemple en détectant automatiquement un risque
        d'allergie ou d'interaction médicamenteuse.\mybox
    \end{exampleblock}
    }
    \vspace{80mm}
\end{frame}
