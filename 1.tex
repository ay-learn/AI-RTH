\section{Introduction}

# Introduction
L’intelligence artificielle est partout, mais elle trouve plus particulièrement
des applications intéressantes dans le domaine de la santé. Les données
médicales constituent une ressource inestimable pour prédire des maladies,
diagnostiquer une pathologie ou améliorer le suivi des patients. Voici donc les
huit dernières avancées en la matière et les perspectives à venir.

Savoir si un grain de beauté est un mélanome, interpréter un scanner cérébral
ou une radiographie des poumons : l'IA est capable de poser un diagnostic
fiable ou de lever des soupçons sur des pathologies. En injectant des milliards
de données dans un programme d'apprentissage, l'algorithme  apprend à «
reconnaître » les signes de la maladie. Google AI, la division de recherche
d'Alphabet, a ainsi mis au point une IA qui prédit le cancer du poumon avec
94,4 % de réussite. Ces procédures permettent aussi d'éviter des tests invasifs
comme des biopsies. L'IA apporte également une aide à la prescription, par
exemple en détectant automatiquement un risque d'allergie ou d’interaction
médicamenteuse.



L’IA, par les algorithmes, aide donc principalement à l’élaboration de
diagnostics. En effet, la machine prescrit le même diagnostic que les médecins
dans 99\% des cas, et dans 30\% des cas, elle propose un traitement plus adapté
que celui des spécialistes. Elle réussit à détecter les cancers du sein dans
89\% des cas, alors que les spécialistes les détectent dans 73\% des cas. 
Ainsi, la robotique étend sa toile dans de nombreux secteurs de la médecine.

# les rôles
le suivi des patients à distance,
les prothèses intelligentes,
les traitements personnalisés grâce au recoupement de données (big data)… 

médecine de précision pour les maladies cardiovasculaires et les cancers;
reconnaissance d’images pour le diagnostic des cancers et des maladies rétiniennes;
conception de médicaments;
amélioration de la prestation des soins de santé;
et plus encore.


# définition 
L’intelligence artificielle est née dans les années 1950 avec l’objectif de
faire produire des tâches humaines par des machines mimant l’activité du
cerveau.

# des probleme philosofique
Les tenants de l’intelligence artificielle dite forte visent à concevoir une
machine capable de raisonner comme l’humain, avec le risque supposé de générer
une machine supérieure à l’Homme et dotée d’une conscience propre. Cette voie
de recherche est toujours explorée aujourd’hui, même si de nombreux chercheurs
en IA estiment qu’atteindre un tel objectif est impossible. 

# 1980
Dans les années 1980, cette approche, dite symbolique, a permis le
développement d’outils capables de reproduire les mécanismes cognitifs d’un
expert. C’est pourquoi on les a baptisés « systèmes experts ». Les plus
célèbres, Mycin (identification d’infections bactériennes) ou Sphinx (détection
d’ictères), s’appuient sur l’ensemble des connaissances médicales dans un
domaine donné et une formalisation des raisonnements des spécialistes qui lient
ces connaissances entre elles pour aboutir à un diagnostic. Les systèmes
actuels, qualifiés d’aide à la décision, de gestion des connaissances ou
d’e‑santé, sont plus sophistiqués. Ils bénéficient de meilleurs modèles de
raisonnement ainsi que de meilleures techniques de description des
connaissances médicales, des patients et des actes médicaux. La mécanique
algorithmique est globalement la même, mais les langages de description sont
plus efficaces et les machines plus puissantes. Ils ne cherchent plus à
remplacer le médecin, mais à l’épauler dans un raisonnement fondé sur les
connaissances médicales de sa spécialité

# developpement en 1980
Cette méthode, née avec le connexionnisme et les réseaux de neurones
artificiels dans les années 1980, se développe aujourd’hui grâce à
l’augmentation de puissance des ordinateurs et à l’accumulation des
gigantesques quantités de données, le fameux big data.

# machine learninig
apprentissage automatique, une méthode fondée sur la représentation
mathématique et informatique de neurones biologiques, selon des modalités plus
ou moins complexes. Les algorithmes d’apprentissage profond (deep learning) par
exemple, dont l’usage explose depuis une dizaine d’années, font une analogie
lointaine avec le fonctionnement cérébral en simulant un réseau de neurones
organisés en différentes couches, échangeant les uns avec les autres. La force
de cette approche est que l’algorithme apprend la tâche qui lui a été assignée
par « essais et erreurs », avant de se débrouiller tout seul. 

# comment fait des AI
Des applications de deep learning existent en traitement d’images, par exemple
pour repérer de possibles mélanomes sur les photos de peau, ou bien pour
dépister des rétinopathies diabétiques sur des images de rétines. Leur mise au
point nécessite de grands échantillons d’apprentissage : 50 000 images dans le
cas des mélanomes, et 128 000 dans celui des rétinopathies, ont été nécessaires
pour entraîner l’algorithme à identifier les signes de pathologies. Pour
chacune de ces images on lui indique si elle présente ou non des signes
pathologiques. A la fin de l’apprentissage, l’algorithme arrive à reconnaître
avec une excellente performance de nouvelles images présentant une anomalie. 


# probleme thecnique
Un problème important soulevé par l’exploitation des données médicales est
relatif à la nature des données disponibles : on estime qu’environ 80 % des
informations sur les patients sont exprimées sous forme de texte libre (comptes
rendus d’hospitalisation ou rapports d’imagerie par exemple). Ces données non
structurées sont notamment caractérisées par une forte variabilité linguistique
qui en complique la manipulation.

# Prévenir plutôt que guérir

# boit noir
Les approches numériques s’apparentent en revanche à une boîte noire, incapable
de justifier ses décisions : nul ne sait ce que fait l’algorithme. Comment, dès
lors, endosser la responsabilité de la décision médicale ? Les données
d’apprentissage sont en particulier biaisées par les préjugés de l’époque et
ceux des concepteurs.

# deep learning
chercheur utilise les algorithmes d’intelligence artificielle pour permettre en
quelque sorte aux ordinateurs d’apprendre par eux-mêmes à explorer le génome à
la recherche de signes avant-coureurs de la maladie. «On appelle cela
l’apprentissage profond ou deep learning en anglais. C’est une des spécialités
de l’équipe de l’Université de Montréal sous la direction de Yoshua Bengio»,
ajoute-t-il. 


# job de exprt

Le travail des médecins radiologistes comprend de nombreuses tâches, dont les
plus rapides comme la détection d'une anomalie ou les plus répétitives comme
les mesures se prêtent bien à l'automatisation.



D'autres tâches, telle la comparaison entre différentes modalités d'imagerie ‒
une échographie antérieure et un examen d'imagerie par résonance magnétique ‒,
requièrent davantage de souplesse intellectuelle. Enfin, les procédures
interventionnelles demandent une dextérité manuelle et un sens commun pour
s'adapter à des situations changeantes.


# Médecine prédictive

Prédire une lésion rénale 48 heures avant qu’elle ne survienne ? C'est possible
grâce à DeepMind, une société d'intelligence artificielle qui a développé un
algorithme capable de détecter des marqueurs biologiques annonciateurs des
lésions rénales. Un diagnostic précoce qui permettrait de réduire de 11 % les
décès dus à ce genre d'événement. L'IA serait aussi capable de prédire la
maladie d'Alzheimer en analysant des images cérébrales ou un échantillon
sanguin, et même des accidents cardiaques en fonction d'un électrocardiogramme
(ECG).

# Nouveaux médicaments
En passant au crible des milliards de molécules, l'intelligence artificielle
est capable de prédire celles qui vont correspondre à un récepteur de cellule
ou d'un virus. En juillet 2019, une équipe australienne a ainsi conçu le
premier vaccin doté d'un adjuvant trouvé par un algorithme. L'IA permet
d'élargir le champ des candidats-médicaments à des molécules que les chercheurs
ne soupçonnaient pas, et de mieux anticiper les effets secondaires des futurs
médicaments.


